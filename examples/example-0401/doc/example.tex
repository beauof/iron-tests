%===============================================================================
%===============================================================================
%
\clearpage
%
\subsection{Example-0401 \texttt{[PLAUSIBLE]}}
%
%===============================================================================
%
\subsubsection{Mathematical model}
%
We solve the Monodomain Equation
%
\begin{align}
    \sigma \Delta V_m(t) = A_m\Big(C_m \dfrac{\partial V_m}{\partial t} + I_{ionic}(V_m)\Big) & &&\Omega = [0, 1] \times [0, 1], \quad t \in [0, 3.0]
\end{align}
%
where $V_m(t)$ is given by the Hodgkin-Huxley system of ODEs \cite{hodgkin1952propagation}

with boundary conditions
%
\begin{align}
    V_m = 0 & &&x = y = 0, \\
    V_m = 0 & &&x = y = 1.
\end{align}
and initial values
%
\begin{equation*}
  \begin{array}{lll}
    V_m(t=0) = -75
  \end{array}
\end{equation*}
%
Additionally a stimulation current $I_{stim}$ is applied for $t_{stim} = [0, 0.1]$ at the center node of the domain (i.e. at $(x,y) = (\frac12, \frac12,)$).
%

Material parameters:
\begin{equation*}
  \begin{array}{lll}
    \sigma = 3.828\\[4mm]
    A_m = 500\\[4mm]
    C_m = 0.58 \quad \text{for the slow-twitch case,} \quad C_m = 1.0 \quad \text{for the fast-twitch case}\\[4mm]
    I_{Stim} = 1200 \quad \text{for the slow-twitch case,} \quad I_{Stim} = 2000.0 \quad \text{for the fast-twitch case}\\[4mm]
  \end{array}
\end{equation*}
%
%===============================================================================
%
\subsubsection{Computational model}
%
\begin{itemize}
    \item{This example uses generated meshes}
    \item{Commandline arguments are:}
        \subitem{number elements X} 		
        \subitem{number elements Y}		
        \subitem{interpolation order (1: linear; 2: quadratic)}
        \subitem{solver type (0: direct; 1: iterative)}	
        \subitem{PDE step size}
        \subitem{stop time}
        \subitem{output frequency} 		
        \subitem{CellML Model URL}
        \subitem{slow-twitch}	
        \subitem{ODE time-step}
    \item{Commands for tests are:}
     \subitem{\verb|./folder/src/example 24 24 1 0 0.005 3.0 1 hodgkin_huxley_1952.cellml F 0.0001|}
     \subitem{\verb|./folder/src/example 24 24 1 0 0.005 3.0 1 hodgkin_huxley_1952.cellml F 0.005|}
     \subitem{\verb|./folder/src/example 10 10 1 0 0.005 3.0 1 hodgkin_huxley_1952.cellml F 0.0001|}
     \subitem{\verb|mpirun -n 2 ./folder/src/example 24 24 1 0 0.005 3.0 1 hodgkin_huxley_1952.cellml F 0.0001|}
     \subitem{\verb|mpirun -n 8 ./folder/src/example 24 24 1 0 0.005 3.0 1 hodgkin_huxley_1952.cellml F 0.0001|}
     \subitem{\verb|./folder/src/example 2 2 1 0 0.005 3.0 1 hodgkin_huxley_1952.cellml F 0.0001|}
     \subitem{\verb|mpirun -n 2 ./folder/src/example 2 2 1 0 0.005 3.0 1 hodgkin_huxley_1952.cellml F 0.0001|}
    \item{This is a dynamic problem.}
\end{itemize}
%
%===============================================================================
%
\subsubsection{Results}
%
\verbatiminput{examples/example-0401/results/results.summary}
\verbatiminput{examples/example-0401/results/failed.tests}
%

\begin{figure}[ht]
  \centering
  \includegraphics[width=0.9\columnwidth]{examples/example-0401/doc/figures/current_run_l1x1_n10x10_i1_s0_p1__t20.png}
  \caption{Result of scenario with $10 \times 10$ elements, $t=20$, direct solver, $p=1$ process}
  \label{example-0401-current-run1-fig}
\end{figure}

\begin{figure}[ht]
  \centering
  \includegraphics[width=0.9\columnwidth]{examples/example-0401/doc/figures/current_run_l1x1_n24x24_i1_s0_p1__t20.png}
  \caption{Result of scenario with $24 \times 24$ elements, $t=20$, direct solver, $p=1$ process}
  \label{example-0401-current-run2-fig}
\end{figure}

\begin{figure}[ht]
  \centering
  \includegraphics[width=0.9\columnwidth]{examples/example-0401/doc/figures/current_run_l1x1_n24x24_i1_s1_p1__t20.png}
  \caption{Result of scenario with $24 \times 24$ elements, $t=20$, iterative solver, $p=1$ process}
  \label{example-0401-current-run3-fig}
\end{figure}

\begin{figure}[ht]
  \centering
  \includegraphics[width=0.9\columnwidth]{examples/example-0401/doc/figures/current_run_l1x1_n24x24_i1_s0_p2__t41.png}
  \caption{Result of scenario with $24 \times 24$ elements, $t=20$, iterative solver, $p=2$ processes}
  \label{example-0401-current-run4-fig}
\end{figure}

\begin{figure}[ht]
  \centering
  \includegraphics[width=0.9\columnwidth]{examples/example-0401/doc/figures/current_run_l1x1_n24x24_i1_s0_p8__t167.png}
  \caption{Result of scenario with $24 \times 24$ elements, $t=20$, iterative solver, $p=8$ processes}
  \label{example-0401-current-run5-fig}
\end{figure}

\begin{figure}[ht]
  \centering
  \includegraphics[width=0.9\columnwidth]{examples/example-0401/doc/figures/current_run_l1x1_n2x2_i1_s0_p1__t20.png}
  \caption{Result of scenario with $2 \times 2$ elements, $t=20$, direct solver, $p=1$ process}
  \label{example-0401-current-run6-fig}
\end{figure}

\begin{figure}[ht]
  \centering
  \includegraphics[width=0.9\columnwidth]{examples/example-0401/doc/figures/current_run_l1x1_n2x2_i1_s0_p2__t41.png}
  \caption{Result of scenario with $2 \times 2$ elements, $t=20$, direct solver, $p=2$ processes}
  \label{example-0401-current-run7-fig}
\end{figure}

With the 'big' target there will be animations created. You get a better understanding of the solutions by looking at them in \verb|iron-tests/examples/example-0401/doc/figures|.

%===============================================================================
%
\subsubsection{Validation}
%
We compare with a Matlab implementation as well as with reference iron files.

The matlab scripts use finite difference discretization instead of finite elements.
The results are qualitatively the same but exactly. The compare script also tests for matlab reference data which is only included for the 2 examples with $24\times 24$ elements. There is a big $L_2$-error. The tolerance is set to a high value to allow for the tests to succeed. With this the comparing-mechanism is tested. Maybe in the future someone succeeds to generate suitable matlab data that then can just be exchanged without having to rewrite the compare script.

The iron files to compare with are the output of the simulation as of Aug. 2017. In that way we can check if the simulation brakes with respect to the current state.
In order to keep file sizes minimal the comparision is only conducted for time steps ${t=0.01, 0.1, 0.2, 1, 2, 3}$ for the 'big' target and ${t=0.1, 0.2, 1}$ for the 'fast' target.
%
%===============================================================================
%===============================================================================
