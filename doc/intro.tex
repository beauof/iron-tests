%===============================================================================
%
\section{Introduction}
%
This document contains information about examples used for testing \iron.
Read: How-to\footnote{\url{https://bitbucket.org/hessenthaler/opencmiss-howto}}
and \cite{OpenCMISS2011}.
%
%===============================================================================
%
\subsection{Cmgui files for cmgui-2.9}
%
%===============================================================================
%
\subsection{Variations to consider}
%
\begin{itemize}
    \item{Geometry and topology}
        \subitem{Dimensions}
        \subitem{Extents}
        \subitem{Number of elements}
        \subitem{Interpolation order}
        \subitem{Generated or user meshes}
        \subitem{quad/hex or tri/tet meshes}
    \item{Initial conditions}
    \item{Load cases}
        \subitem{Dirichlet BC}
        \subitem{Neumann BC}
        \subitem{Volume force}
        \subitem{Mix of previous items}
    \item{Sources, sinks}
    \item{Time dependence}
        \subitem{Static}
        \subitem{Quasi-static}
        \subitem{Dynamic}
    \item{Material laws}
        \subitem{Linear}
        \subitem{Nonlinear}
    \item{Material parameters, anisotropy}
    \item{Solver}
        \subitem{Direct}
        \subitem{Iterative}
    \item{Test cases}
        \subitem{Numerical reference data}
        \subitem{Analytical solution}
    \item{A mix of previous items}
\end{itemize}
%
%===============================================================================
%
\subsection{Folder structure}
%
TBD..
%
%===============================================================================
