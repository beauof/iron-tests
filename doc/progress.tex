%===============================================================================
%
\clearpage
%
\section{Progress}
%
People working on setting up tests in alphabetical order (surnames) with initials:
%
\begin{itemize}
    \item CB : Christian Bleiler
    \item  NE : Dr.-Ing.\ Nehzat Emamy
    \item  AH : Andreas Hessenthaler
    \item  TK : Thomas Klotz
    \item  AK : Aaron Kr\"amer
    \item  BM : Benjamin Maier
    \item  SM : Sergio Morales
    \item  MM : Mylena Mordhorst
    \item  HS : Harry Saini
\end{itemize}
%
\subsection{Equations to test}
%
Test single-physics problems before multi-physics problems!
%
\begin{itemize}
    \item Diffusion equation (Laplace, Poisson, Generalized Laplace, ALE Diffusion, etc.)
    \item Linear elasticity equation (compressible and incompressible)
    \item Finite elasticity equation (compressible and incompressible Mooney-Rivlin, etc.)
    \item Navier-Stokes equation (ALE, Stokes, etc.)
    \item Monodomain equation
    \item CellML models
    \item Skeletal muscle models
    \item Fluid-structure interaction
    \item etc.
\end{itemize}
%
\subsection{Setting up a new test}
%
Use the following guideline to set up a new test:
%
\begin{enumerate}
    \item Check if it is already there
    \item Talk to other developers
    \item Create a new subfolder examples/example-0xxx
    \item Document the setup (computational domain, etc.) in examples/example-0xxx/doc/example.tex
    \item Set up example with all parameters as command line arguments, see Section~\ref{variations-sec}
    \item Set up reference results (CHeart, Abaqus, analytical solution, etc.)
    \item Set up script to run all tests in your example directory
    \item Set up script to perform comparison between iron results and reference results
    \item Set up visualization scripts
    \item Compile, run, test, visualize your example
    \item Compile, run, test, visualize all examples
\end{enumerate}
%
For each example, progress is documented in the respective section titles
with the following \texttt{\color{red} TAG}:
%
\begin{itemize}
    \item{\texttt{\color{red} DOCUMENTED}: finish the documentation of the example (spatial domain, number of time steps, boundary conditions, etc.}
    \item{\texttt{\color{red} COMPILES}: example compiles (for default parameters)}
    \item{\texttt{\color{red} RUNS}: example runs (for default parameters)}
    \item{\texttt{\color{red} CONVERGES}: no convergence issues (for default parameters, results not plausible)}
    \item{\texttt{\color{red} PLAUSIBLE}: results look sensible (for default parameters)}
    \item{\texttt{\color{red} VALIDATED}: for all parameter sets it gives the correct results as compared to CHeart/Abaqus/analytical solution (includes visualization scripts, run scripts, comparison scripts, documentation!, \ldots)} 
\end{itemize}
%
Move all tags \texttt{\color{red} CONVERGE, PLAUSIBLE} to
\texttt{\color{red} VALIDATED}.\\

\noindent Next steps include:
%
\begin{itemize}
    \item Everybody runs everything!
    \item Meeting with Oliver
    \item Meeting with Auckland
\end{itemize}
%
\subsection{Long-term goals}
%
\begin{itemize}
    \item Different testing targets
    \subitem \texttt{SMALL} : small, fast tests
    \subitem \texttt{BIG} : same as before; further, bigger and more complex geometries, convergence analysis
    \subitem \texttt{PARALLEL} : same as before but in parallel
    \item Add more examples/those which were on the agenda but not started
    \item Jenkins continuous testing, integration and deployment
    \subitem test \texttt{SMALL/BIG/PARALLEL} targets
    \subitem integrate with GitHub (pull-requests triggers Jenkins, merge on success)
\end{itemize}
%
%===============================================================================
